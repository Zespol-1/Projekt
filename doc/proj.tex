\documentclass[]{article}
\usepackage[T1]{fontenc}
\usepackage{amsfonts}
\usepackage{graphicx}
\usepackage{float}
\usepackage[polish]{babel} % English language/hyphenation
\usepackage[linesnumbered,lined,boxed,commentsnumbered,ruled,vlined]{algorithm2e}

%opening
\title{Projekt Zespołowy \\
	\Huge Etap projektu – projektowanie rozwiązania na zadaną architekturę}
\author{\\ \\ \\ Autorzy:
	\\Biernacka Kamila\\ 
	Kania Dominik\\ 
	Leśniak Mateusz\\ 
	Maziarz Wojciech\\ \\ \\ \\ \\ \\ \\ \\ \\ \\ \\ \\ \\ \\ \\ \\ \\  }
\date{kwiecień 2021}

\begin{document}

\maketitle
\newpage



\begin{abstract}

Poniższe sprawozdanie jest wynikiem naszej pracy na drugim etapie projektu zespołowego z implementacji metody indeksu w architekturach GPU. Przedstawimy w nim przygotowane przez nas projekty i rysunki koncepcyjne wymaganych do zaimplementowania algorytmów.

\tableofcontents
\newpage

\end{abstract}

\section{Mnożenie modularne dużych liczb}
	
\section{Poszukiwanie relacji i faktoryzacja w bazie}
	\subsection{Szybkie potęgowanie modularne}
	Metoda indeksu wymaga obliczenia wartości typu $a^{b}\;{mod}\;n$.
	Szybkie potęgowanie modularne jest prostym algorytmem pozwalającym zredukować liczbę mnożeń i dzieleń modulo z $b$ do \textit{O(log b)}.
	
	\begin{algorithm}[H]
		\caption{\texttt{szybkie potęgowanie}} \label{szybkie_pot} 
		\SetKwInOut{Input}{Input}\SetKwInOut{Output}{Output}
		
		\Input{podstawa potęgi $a$, wykładnik potęgi $b$, modulnik $n$}
		\Output{$a^{b}\;mod\;n$}
		\BlankLine
		$bits=to\_bin(b)$\\
		$nbits=length(bits)$\\
		$a=a\%n$\\
		$result=1$\\
		$x=a$\\
		
		\For{i=0 to nbits}{
			\If{bits[i]==1}{
				$result=result*x$\\
				$result=result\%n$\\
			}
			$x=x*x$\\
			$x=x\%n$\\
		}
		\Return{$result$}
	\end{algorithm}
	
	\subsection{Fakoryzacja w bazie}
	
	\subsection{Budowa relacji}

\section{Eliminacja Gaussa w pierścieniu \(\mathbb{Z}_{p-1}\)}
	\subsection{Algorytm Euklidesa}

	\subsection{Rozszerzony algorytm Euklidesa}
	
\end{document}
